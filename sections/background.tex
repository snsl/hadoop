\label{background}
This section gives a brief overview of the main components of the
approach proposed in this work - MapReduce and object-based storage.

\subsection{MapReduce}
\label{mapreduce}
MapReduce is a parallel computational model developed originally by Google~\cite{Dean:2008:MSD:1327452.1327492}
and it is widely used for distributed processing of large datasets over clusters. Data in MapReduce is
represented with $<$key, value$>$ pairs. The first step of an
application using MapReduce is to partition its input data
into blocks that are replicated across datanodes. This data is then
processed in parallel with mappers that
produce intermediate data from the input data. This intermediate data
is then fed to reducers which process
the intermediate data based on intermediate keys and combine intermediate values to form the final output data of
the application.

Hadoop~\cite{apache_hadoop} is a commonly used open-source implementation of MapReduce and it consists of
two layers - storage and computation. The MapReduce algorithm is implemented in the computational layer, whereas
the storage layer is managed by the Hadoop Distributed File System (HDFS). HDFS provides redundancy by replicating data
three times (by default) across the storage nodes while also trying to
preserve the data locality of the system. One
replica is stored locally, the second replica is located in another node in the same rack and the last replica is
stored in another rack. Hadoop applications also follow a write-once-read-many workflow and as a
result, they can benefit from the approach presented in this paper extensively, as data is not ingested
from a remote storage cluster to the compute cluster.

\subsection{Object-Based Storage}
Object-based storage is a storage model that stores and accesses data in flexible-sized logical
containers, called~\textit{objects}, instead of using the traditional fixed-sized, block-based
containers. Objects store metadata either together with data or in dedicated object attributes. Metadata can
be any type of data (i.e. size, access permissions, creation time etc.) describing the actual object data.
Increasing interest in object-based storage led to the standardization of the T10 object-based storage interface~\cite{osd3}.
There have been many examples of object-based storage systems in cluster file systems; such as PVFS~\cite{Carns:2000:PPF:1268379.1268407}
and Lustre~\cite{lustre_web}, as well as scaled out cloud storage systems; such as Ceph~\cite{cephorig},
OpenStack Swift~\cite{openstack_swift}, and Amazon S3~\cite{amazon_s3}. These systems are typically
designed as a software interface on top of an existing file system.
